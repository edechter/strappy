\documentclass{article}
\usepackage{amsmath}
\usepackage{amssymb}

\usepackage{tikz}
\usepackage{tikz-qtree}
\usepackage{subcaption}
\usetikzlibrary{arrows,automata, positioning, patterns}

\begin{document}
\begin{tikzpicture}
\node {}
child { node [minimum size=0pt] {}
  child { node (S) {S} }
  child { node (*) {*} }
  }
child {node (I) {I}}
 ;

\end{tikzpicture}

\begin{tikzpicture}
\node {}
child { node  {$\lambda x$}
  child { node (*) {$*$} }
  child { node (x) {$x$} }
  }
child {node (x) {$x$}}
 ;

\end{tikzpicture}

\begin{align}
E \rightarrow \;  &( E \; E ) \\
 & |\; S \;|\; B \;| \;C \;| \;I \\ 
 & |\; c_i \in \text{ primitives }
\end{align}

\begin{align}
T \rightarrow \;  &( T \mapsto  T ) \\
 & |\; \alpha \;|\; \beta \;|\; \dots \text{ (type variables) }\\
 & |\; Int \;|\; Bool \;| \; Real \;| \; etc.
\end{align}


\textbf{language} $\mathcal{L}$\\
 
\textbf{tasks} $\{t_k\}_{k=1}^K$\\

$t_k \; : \mathcal{L} \rightarrow \{0, 1\}$\\

$e \in \mathcal{L} \text{\textbf{ solves }} t_k \text{ if } t_k(e) =
1$\\

\textbf{goal} solve as many tasks in T as possible

\begin{align}
\mathbf{S}\; f\; g\; x\; &\rightarrow (f\; x)\; (g\; x)\\ 
\mathbf{K}\; f\; x\; y\; &\rightarrow (f\; x)\\ 
\mathbf{I}\; x &\rightarrow  x & \text{ (identity) }
\end{align}

$\lambda x. \; * \; x \; x$\\\\

$S * I $\\

$(+ \; 1)\; : \; Int \rightarrow Int$\\

$NOT \; : \; Bool \rightarrow Bool$\\

$\sigma \rightarrow Int$\\

$\sigma \rightarrow \sigma$\\


\textbf{primitive combinators } $C = c_1, \dots, c_N$\\

\textbf{associated probabilities } $D = p_1, \dots, d_N, p_i \in [0,1], \sum_i p_i = 1$\\

set of $c$'s that unify with $\tau(c_n)$\\


${ p(e)~=~\prod_{c \in C_e} p(c | \tau(c)) }$\\

$p(c_n| \tau(c_n)) \propto
\frac{p_n}{\sum_{c_j \in C_{\tau(c_n)}} p_j}$\\

$p(c | \tau(c))$: probability of using primitive $c$ when
the requesting type is $\tau(c)$. In turn, we define the conditional
probability for each combinator $c_n$ to be $p(c_n| \tau(c_n)) \propto
\frac{p}{\sum_{c_j \in C_{\tau(c_n)}} p_j}$.  That is, $p(c_n | \tau(c_n)$ is
proportional to the probability of sampling $c_n$ from the multinomial
distribution defined by the $p_n$'s, conditioned on the requesting
type, with the constant of proportionality chosen to ensure that the\\

\textbf{frontier} $\mathcal{F}$ set of expressions considered\\

\textbf{frontier size} $N$ size of frontier\\

a task $t$ is \textbf{hit} by $\mathcal{F}$ if $e \in \mathcal{F}$ s.t. $e$ solves $t$\\

\begin{itemize}
\item \textbf{explore} the frontier
\item \textbf{compress} solutions to hit tasks
\item repeat
\end{itemize}

number of unique subtrees in $e$\; $| e |$\\

\begin{itemize}
\item no application should appear more than once in the grammar
\item every rule should be used more than once
\item estimate the $p_i$ for each grammar element by counting
\end{itemize}
 







 
\end{document}


