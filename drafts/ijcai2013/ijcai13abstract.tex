\documentclass{article}

\usepackage{ijcai13}
\usepackage{times}
\usepackage{amsmath} 
\usepackage{latexsym} 
\usepackage{ dsfont }
\usepackage{tikz}
\usepackage[ruled,vlined,linesnumbered]{algorithm2e}
\usetikzlibrary{arrows,automata}

\DeclareMathOperator*{\argmax}{arg\,max} 

\title{The E-C Algorithm: conceptual bootstrapping via grammar-based
  program compression.}

%%\title{The Exploration-Compression Algorithm: how to learn programs by discovering useful subroutines.}
\author{Eyal Dechter \\
MIT\\
USA \\
edechter@mit.edu
\And
Josh Tenenbaum \\
MIT\\
USA \\
jbt@mit.edu
\And 
Ryan Adams \\
Harvard University\\
USA \\
rpa@seas.harvard.edu}

\begin{document}

\maketitle

\begin{abstract}
Suppose a learner is faced with a set of problems about which it knows
nearly nothing: it doesn't know the distribution of problems, or what
an appropriate loss function might be. For each hypothesized solution
to a problem it gets one bit: right or wrong. How can such a learner
every get off the ground? The key is to find, by brute force, a
solution to one or more simple problems in that domain and use the
information thus gained to pursue subsequent searches in an informed
manner. Here, we formalize this intuition, where the solution space is
the space of typed functional programs and the gained information is
stored as a stochastic grammer over programs. Specifically, we propose
the E-C algorithm, a two step iterative algorithm for exploring such
spaces: in the E step, the learner \emph{explores} a finite subset of
the domain for solutions, guided by a stochastic grammar; in the C
step, the learner \emph{compresses} the successful solutions from the
E step to estimate a new stochastic grammar. As we show in several toy
domains, iterating these two steps quickly propels the learner from a
state of solving almost no problems to solving a large number of
problems by gaining abstract knowledge of the structure of the
solution space.
\end{abstract}

\end{document}
