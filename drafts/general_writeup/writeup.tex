\documentclass{article}

\usepackage[ruled,vlined,linesnumbered]{algorithm2e}
\usepackage{amsmath} 

\usepackage{ amssymb }
\usepackage{listings}
\lstloadlanguages{Haskell}
\lstnewenvironment{code}
    {\lstset{}%
      \csname lst@SetFirstLabel\endcsname}
    {\csname lst@SaveFirstLabel\endcsname}
    \lstset{
      basicstyle=\small\ttfamily,
      flexiblecolumns=false,
      basewidth={0.5em,0.45em},
      literate={+}{{$+$}}1 {/}{{$/$}}1 {*}{{$*$}}1 {=}{{$=$}}1
               {>}{{$>$}}1 {<}{{$<$}}1 {\\}{{$\lambda$}}1
               {\\\\}{{\char`\\\char`\\}}1
               {->}{{$\rightarrow$}}2 {>=}{{$\geq$}}2 {<-}{{$\leftarrow$}}2
               {<=}{{$\leq$}}2 {=>}{{$\Rightarrow$}}2 
               {\ .}{{$\circ$}}2 {\ .\ }{{$\circ$}}2
               {>>}{{>>}}2 {>>=}{{>>=}}2
               {|}{{$\mid$}}1               
    }

\DeclareMathOperator*{\argmax}{arg\,max} 

\begin{document}

\newcommand{\learner}{\emph{Learner }}
\newcommand{\evaluator}{\emph{Evaluator }}



\section{The Learning Problem}
We want to characterize the general task of the learner, whom we will
call Linda. At various intervals, Linda is provided with sensory data
$x$. In reponse, she is given the opportunity of submitting a program
$\phi$ to an evaluator, henceforth named Dr. Eval. Dr. Eval produces a
reward $R(\phi, x; \bar{y})$ which he delivers back to Linda, where
$\bar{y}$ simply denotes any other variables Linda is not aware of
that influences Dr. Eval's reward. Linda processes the information,
hopefully learning from her experience with Dr. Eval.

What kinds of scenarios can this learning paradigm (naturally)
contain? Most basically, Linda might be a simple organism: the data
$x$ might be local chemical gradients, the program $\phi$ the motor
program Linda will execute upon sensing $x$, and Dr. Eval might return
the cumulative effect upon Linda of executing that motor program. Note
that in this case $\phi$ does not take $x$ as an input. It has no
input; it is merely a motor program built anew for each sensory datum
that Linda receives.

Of course, we might want Linda, the still humble organism, to be
capable of learning a higher-level planning procedure. Such a
procedure would be a program that, given sensory data, would produce a
motor program. In our framework, this would be represented by letting
$x$ stand for a list of data (maybe several days worth of discrete
food finding missions) and letting $\phi$ be the planing
procedure. Dr. Eval would accordingly take $\phi$, apply it to each
datum in $x$ and return a cumulative reward of some sort. We will
defer for now the problem of deferred reward, in which most learners
do not receive immediate signals about the utility of their actions.

Moving upward somewhat in the hierarchy of intelligent beings,
consider Linda the electrical engineer. She is requested to construct
a circuit $\phi$ to a certain specification $x$. Dr. Eval, her
professor tests the circuit against the requested specification and
gives her a grade. Or perhaps she doubles as the evaluator. As in the
case of Linda the simple organism, we can conceive of $\phi$ as the
general circuit creation procedure rather than the circuit, and create
a higher-level learning problem.

A different perspective on this learning problem is one that is
removed from this supervised reinforcement paradigm: instead of
acheiving a reward of some sort, we can imagine that Linda wants to
find the program $\phi$ which generated the data $x$. Here, Dr. Eval
is the world or some simulator of the world and a comparator which
signals to Linda whether she has successfully simulated the data. This
paradigm corresponds to a deterministic version of the generative
model learning task common these days in probabilistic machine
learning and AI.

TODO: concept learning
TODO: physicist (or scientist in general) 


\section{A Universal Basis for Computation}
Our goal is to explain the minimal conditions necessary for concept
learning, and to do so we posit a minimal model of
computation. Various universal models of computation have been
proposed; the most famous is the Turing machine. Though the Turing
machine is an intuitive abstract machine, and though its construction
corresponds to various parts of real-world computing machines,
formally simpler models of computation have been proposed. Most
importantly is the Lambda calculus, proposed by Church, and the basis
of modern functional programming languages. The Lambda calculus is a
rewrite system that is provably equally powerful to Turing
machines. Despite its popularity, the implementation of the Lambda
calculus is complicated by the presence of variables bound by the
abstracting $\lambda$s. The necessary relationships between variables
and their binding $\lambda$s means that the syntax of the Lambda
calculus is not entirely modular. That is, if we represent an
expression in the lambda calculus as a tree, then every subtree of this
expression tree is not itself a valid expression tree in the langauge. 

An alternative to the $\lambda$ calculus we use the variable-free
model of computation known as Combinatory Logic (CL). CL was
introduced by Shonfinkel in order to avoid of the mathematical
complexities of $\lambda$-calculus. The CL has a simpler syntax than
$lambda$-calculus: a combinator $C$ is merely the application of two
other combinators $ C = A B$. The semantics of the language is defined
by the primitive combinators chosen for the particular instance of
$CL$. It is well known that a universal basis of two primitive
combinators exists, the $SK$ basis. However, expressions in CL are
much more readible if we extend this primitive basis with a few other
high order combinators. Namely, we include identity $I$, and two
composition combinators $B$ and $C$.

\section{History and related literature}
Angluin 1988~\cite{angluin1988queries}: investigates the complexity of
learning concepts that are subsets of a hypothesis space via a number
of primitive queries such as membership, equivalence, subset, etc. \\

Valiant 1984~\cite{valiant1984theory}: original paper on Valiant's
computational learning theory.

Shapiro 1981~\cite{shapiro1981algorithm}: on the model inference
problem. Focused on learning programs or axioms from a set of positive
and negative examples. Explicitly related to Kuhnian philosophy of
science.

Inferring a program for computing a recursive function: 
Gold 65~\cite{gold1965limiting}, 
Gold 67~\cite{gold1967language},
Gold 78 ~\cite{gold1978complexity},
Blum  Blum 75~\cite{blum1975toward}

Inductive Logic Programming (ILP): ILP has a rich history and
literature, and in some sense its goals are quite close to ours. In
recent years, Genetic Programming style techniques have been applied
to ILP; see~\cite{divina2006evolutionary}.

\section{Type Theory}
In using type theory to facilitate program induction we will mostly be
concerned with using the types of subprograms to determine how
subprograms can be glued together. This focus is nicely stated by Girard~\cite{girard1989proofs}:

\begin{quote}
At a more general level, abstracting away from any peculiar syntactic
choice, one should see a type as an instruction for \emph{plugging}
things together. let us imagine that we program with \emph{modules},
\emph{i.e.} closed units, which we can plug together. A module is
absolutely closed, we have no right to open it. We just have the
ability to use it or not, and to choose the manner of use
(plugging). The type of module is a of course completley determined by
all the possibling \emph{pluggings} it allows without crashing. In
parcticular, one can always susbtitute a module with another of the
same type, in the vent of a breakdown, or for the purpose of
optimisation.
\end{quote}

\section{Domains for concept learning}

\section{Coding Theory}
Consider a distribution $P(\cdot)$ over strings in some set $X \in
\{0, 1\}^*$. Our goal is to find a code $D$ that minimizes the
expected length of a sample from $X$.

\section{Learnable Curriculum}
We would like to formally characterize when a program can be learned
that represents elements in a given dataset. In general such programs
always exist; we can simply iterate over all recursive functions until
we find a set that evaluates to each element in the dataset. However,
such a search will usually take a prohibitively long time, so we wish
to restrict our search to short programs.  Thus, we say that a datum
is learnable if a sufficiently short program exists that evaluates to
it. This is simply the statement that the Kolmogorov complexity
$K_U(x)$ of a datum $x$ (relative to our universal computer $U$) is
less than some fixed number of bits $r$. 

Moreover, we should be able to use previously learned programs to help
us search for new programs. This phenomenon, in which a previously
learned datum is used to represent a later learned datum, we will term
\emph{bootstrapping}. We say that $x$ \empth{bootstraps} $y$ if the
conditional Kolmogorov complexity $K(y|p_x) < r$ when $U(p_x) = x$ and
$r$ is again our resource bound. 

Let $D= \{d_1, \dots, d_N\}$ be a \emph{dataset}. A \emph{curriculum}
$C$ for $D$ is an ordered tuple $C = (c_1, \dots, c_M)$ of elements in
$D$. Not all data $d_i$ must appear in $C$, and each $d_i$ can appear
more than once.

\section{Model of Computation: Combinatory Logic}
The representational language and related computational model used to
implement our learner is a typed Combinatory Logic
(CL)~\cite{hindley1972introduction}. CL is symbolic model of
computation that is similar to the more popular $\lambda$-calculus,
but, for our purposes, has one very significant difference: it does
not contain variables. $\lambda$-calculus has three basic syntactic
features -- variables, applications, and abstractions -- that
correspond appealingly with intuitions from mathematics and functional
programming. But besides the semantic difficulties that variables
cause (documented elsewhere), variables additionally complicate the
modularity and compositionality of expressions written in
$\lambda$-calculus. Program induction requires the efficient search
over expressions in the chosen model of computation; being able to
treat most subexpressions of well-formed expressions as themselves
well-formed expressions makes things easier. This was first noticed by
Briggs and O'Neill in using combinators for genetic
programming~\cite{briggs2006functioinal}. 

Example: Consider the function $f = \lambda y. \lambda x. +\,
(*\, x\, x) (*\, x\, x)$ (note that all functions we are considering
will be \emph{curried}; that is, all functions will be functions of
only one argument, and functions of multiple arguments will be
represented by higher-order functions of a single argument such that
the application of the function to an argument returns a function
whose argument is the next argument in the original series of
arguments.)



\section{Intuitive Psychology}
How can we express the knowledge and relationships of intuitive
psychology as a set of expressions in combinatory logic? To simplify
this task, we will attempt to express these relationships in Haskell,
a sugared and polymorphically typed lambda calculus, knowing that any
expression in this lambda calculus can be systematically translated
into combinatory logic. Most importantly, any modularization and
compositional structure present in the Haskell code will remain
present in its CL translation.

Let's imagine the following situation. You observe several agents
(say, Josh, Max, and Chris) in various scenarios in which they choose,
from among the various options, one of three bars (say, a Zone bar, a
Luna bar and a dark chocolate bar). We will try to represent our
knowledge about these agents, their propensities, and the properties
of these bars in Haskell, as a demonstration that such a
representation is both possible in this langauge, and can be somewhat
natural.

Let's first define the primitive types and combinators at our disposal:


\begin{lstlisting}[mathescape]
-- | Primitive Types
-------------------------
Int -- integers 1, 2, etc.
Bool -- {True, False}
Person $\triangleq$ Int
Object $\triangleq$ Int
Scenario  $\triangleq$ (Person, [Object])

-- | Primitive Combinators
------------------------
-- List Combinators:
map :: (a -> b) -> [b] -> [a]
foldl1 :: (a -> a -> a) -> [a] -> a
elem :: a -> [a] -> Bool
head :: [a] -> a
tail :: [a] -> [a]
case :: [(Bool, a)] -- a set of boolean tests and the result if True
        -> a -- the default result (else clause) 
        -> a
(:) :: a -> [a] -> [a] -- cons
(++) :: [a] -> [a] -> [a] -- concatenate

-- Boolean Combinators
If _ Then _ Else :: Bool -> a -> a
And :: Bool -> Bool -> Bool
Or ::  Bool -> Bool -> Bool
Not ::  Bool -> Bool -> Bool

-- Arithmetic Combinators
(>) :: Int -> Int -> Bool
(<) :: Int -> Int -> Bool
(==) :: Int -> Int -> Bool
(+) ::  Int -> Int -> Bool
(-) :: Int -> Int -> Bool
1, 2, $\dots$, 10 :: Int -- | the integers, 1 through 10

-- Givens
------------

Josh :: Person -- | this is just shorthand, i.e., Josh would be mapped
                  to some integer, say, 1, Max to 2, etc..
Max :: Person
Chris :: Person

ZoneBar :: Object
LunaBar :: Object
ChocolateBar :: Object

-- Learned representation
-------------------------
choose' :: Scenario -> Object
choose' (person, objs) = max' (map (util' person) objs)

max' :: [Int] -> Int
max = foldl1 (>)

util' = Person -> Object -> Int
util' person obj = case (utilityCases' person obj) 0 -- | 0 is the default utility in this case

utilityCases' :: [Person -> Object -> (Bool, Int)]
utilityCases' = \p -> \o -> (
                     \p -> \o -> (isHealthyPerson p, healthyPersonCalculus o)



\end{lstlisting}


\section{Algorithm 0}
In order to motivate our approach, we present the following problem:
suppose we have a probability model $P(D, E, \mathcal{L}) = P(D|E)
P(E|L) P(L)$ where $L$ represents the parameters of a latent model,
$E_i$ are drawn $i.i.d$ from $P(\dot | L)$ and $D_i$ are drawn $i.i.d$
from $P(\dot|E_i)$. The $E_i$ are hidden, the $D_i$ observed. Suppose
we are able to 1) draw sampled of $E_i$ given $L$ and 1) we can
enumerate $E_i$ in order of decreasing probability under $L$. Our goal
is to determine the most likely $L$ given the data $D$. A rejection
sampling algorithm approach might involve sampling the $E_i$ from the
current value of $L$, accepting $E_i$ if  To solve this problem, we propose
Algorithm~\ref{alg:short}: at each iteration of our algorithm, we
choose

\begin{minipage}{.7\linewidth}
  \begin{algorithm}[H]
    \SetAlgoLined 
    \KwData{$\mathcal{D}=(D_1, \dots, D_M)$,
      $\mathcal{L}^{t-1}$ , $0 \geq K \leq 1$} 
    \KwResult{updated library $\mathcal{L}^t$}
    $E \leftarrow \{ E_i : p(E_i\, |\, \mathcal{L}^{t-1}) > K\}$\;
    $S \leftarrow \{ h \in H \cap D \}$\;
    $\mathcal{L}^t \leftarrow \argmax_L p(L \, | \, S)$
    \caption{Algorithm 0}
    \label{alg:short}
  \end{algorithm}
\end{minipage}


\section{Algorithm}
In this section, I describe the program learning algorithm. The
algorithm is parameterized by the following parameters: 

\begin{itemize}
\item depth $d$: this bounds the search over expressions at each
  iterations to expressions of depth $\leq d$;
\item log likelihood search bound $l$: this bounds the search over
  expressions at each iteration to expressions whose log likelihood is
  $\geq l$;
\item library prior $\mathcal{L}^0$: this data structure contains the
  primitive combinators and associated pseudo-counts that represent
  the background knowledge of the algorithm. This should also include
  any necessary base combinators (e.g. S, K, I, etc.).
\end{itemize}

The inputs to the algorithm are:
\begin{itemize}
\item a dataset $\mathcal{D} = (D_1, \dots, D_M)$, each of which is an
  expression in the language;
\item a evaluator $\mathcal{E}$ which is a function taking a datum
  $D_i$ and an associated expression $\phi_i$ and returns a reward
  signal $\mathcal{E}(D_i, \phi_i)$.
\end{itemize}

\begin{minipage}{.7\linewidth}
  \begin{algorithm}[H]
    \SetAlgoLined 
    \SetKwFunction{EnumerateCombinators}{EnumerateCombinators}
    \SetKwFunction{IsSuccessful}{IsSuccessful}
    \SetKwFunction{filter}{filter}
    \SetKwFunction{FindBestNewCombinator}{FindBestNewCombinator}
    \SetKwFunction{ReEstimateLibrary}{ReEstimateLibrary}
    \KwData{$\mathcal{D}=(D_1, \dots, D_M)$,
      $\mathcal{E}$, $\mathcal{L}^{t-1}$, $d$, $l$} 
    \KwResult{updated library $\mathcal{L}^t$}
    $E \leftarrow $ \EnumerateCombinators{$d$, $l$, $\mathcal{L}^{t-1}$}\;
    $E' \leftarrow $ 
      \filter{$\lambda e.$ \IsSuccessful{$\mathcal{E}$, $\mathcal{D}$, $e$}, $E$}\;
    $C \leftarrow$ \FindBestNewCombinator{$E'$}\;
    $\mathcal{L}^t \leftarrow$ \ReEstimateLibrary{$\mathcal{L}^{t-1}$, $C$, $E'$}\;
    \caption{OneStep}
  \end{algorithm}
\end{minipage}


\section{Language Implementation}
I've found various resources to guide implementation of typed
languages.  A simple guide to the implementation of dependently typed
programming languages is found in Loh et al.~\cite{lowsimply}.

\section{Holes}
In real-world applications, many of the useful degrees of freedom are
continuous, and it would be useful to have parameters that could be
fit in a systematic way. Doing this requires a small extension to the
framework presented so far: we add a special type of construct as a
primitive of the language. We call these holes. Each hole is a
placeholder for values of a specific type, within a specific
domain. For example, a continuous hole is bounded within a continuous
integer and has the type of a real value.

A combinator with no holes is filled.







\bibliographystyle{plain}
\bibliography{writeup}

\end{document}
