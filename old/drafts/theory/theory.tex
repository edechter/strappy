\documentclass{article}
\usepackage{amsmath}
\begin{document}

Consider the problem of learning boolean functions of $N$ variables.
Suppose we are given an oracle that, when handed such a function, tells us if it is correct or not.
Then, the size of the space of possible functions is $2^{2^N}$.

Supppose there exists a factorization of the target function, called a \emph{curriculum}.
The \emph{curriculum} is a set of $C$ boolean functions of arity $d$ such that the target function may be written only in terms of up to $d$ members of the set consisting of the union of the $N$ variables and the $C$ curriculum functions.

Now we'll calculate an upper bound on the work required to identify any such factorizations.
First, there must be at least one curriculum function that is only a set of the $N$ variables.
There are $\binom{N}{d}$ ways of choosing the input variables and $2^{2^d}$ ways of constructing functions from those variables, so there are $\binom{N}{d}2^{2^d}$ different functions to consider.
If we are given corresponding oracles for all $C$ members of the curriculum, then we will need at most $C\binom{N}{d}2^{2^d}$ queries to such oracles.
Similarly, the curriculum function which is a function of $d$ members of the $N$ variables and the previous curriculum function requires at most $(C-1)\binom{N+1}{d}2^{2^d}$ queries.
Continuing this reasoning, we get the following upper bound on the number of oracle queries required to identify the function of interest given the curriculum:
$$
\binom{N+C}{d}2^{2^d} + \sum_{j=0}^{C-1} (C-j) \binom{N+j}{d} 2^{2^d}
$$
where the first term comes from the oracle queries required to reconstruct the taraget function after learning the curriculum.

This analysis applies in the case where we don't know what order to learn the curriculum in. This might be unrealistic for human learners.
However, it is the setting EC uses.

One problem with the previous analysis is that it does not reflect the fact that there are multiple factorizations, and for each of them, the value of $d$ is a function of $C$.
One nice property of the analysis is that it gives us an upper bound on what $d$ should be given $C$: it should be less than the value which makes the bound equal to $2^{2^N}$, because otherwise, the curriculum doesn't help us.

We'd like to show that the reduction in the search space, for certain classes of factorizations, is large (exponential?) in $C$.
One approach would be to analyze the change in $d$ when $C$ is decreased by one.
If $C'=C-1$ then $d\leq d'\leq 2d-1$.
But, most of the nodes don't have their degree changed, so it seems that we ought to change out analysis to reflect that fact.

Another (probably more difficult) problem to analyze is that of discovering the curriculum as one learns.
This is closer to what EC does.
\end{document}